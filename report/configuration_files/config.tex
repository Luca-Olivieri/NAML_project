% Define blue color typical of polimi
\definecolor{bluepoli}{cmyk}{0.4,0.1,0,0.4}

% Define custom math operators

\DeclareMathOperator*{\argmax}{argmax}
\DeclareMathOperator*{\argmin}{argmin}

\newcommand{\subtitle}[1]{\par\vspace{0.5\baselineskip}\noindent\textcolor{bluepoli}{\textbf{#1}}\par}
\newcommand{\nb}{\textbf{N.B.}\\}

\newcommand{\colspace}[1]{\text{Col}\paren{#1}}
\newcommand{\rowspace}[1]{\text{Row}\paren{#1}}
\newcommand{\nullspace}[1]{\text{Null}\paren{#1}}

\renewcommand{\dim}[1]{\text{dim}\paren{#1}}
\renewcommand{\rank}[1]{\text{Rank}\paren{#1}}

\newcommand{\vect}[1]{\mathbf{#1}}
\newcommand{\matr}[1]{\mathbf{#1}}
\newcommand{\estim}[1]{\hat{#1}}

\newcommand{\realset}{\mathbb{R}}
\newcommand{\complexset}{\mathbb{C}}
\newcommand{\smoothset}{\mathcal{C}}
\newcommand{\loss}{\mathcal{L}}
\newcommand{\normal}[1]{\mathcal{N}\paren{#1}}

% Commands for time-series

\newcommand{\autocov}[2]{\gamma_{#1}\paren{#2}}
\newcommand{\autocorr}[2]{\widetilde{\gamma}_{#1}\paren{#2}}
\newcommand{\spectrum}[2]{\Gamma_{#1}\paren{#2}}
\newcommand{\complspec}[2]{\Phi_{#1}\paren{#2}}
\newcommand{\fourier}[1]{\mathcal{F}\bracket{#1}}
\newcommand{\dirac}[1]{\delta\paren{#1}}
\newcommand{\res}[1]{\varepsilon\paren{#1}}
\newcommand{\wn}[1]{\mathcal{WN}\paren{#1}}
\newcommand{\AR}{\text{AR}}
\newcommand{\MA}{\text{MA}}
\newcommand{\ARMA}{\text{ARMA}}
\newcommand{\ARMAX}{\text{ARMAX}}

\newcommand{\neper}{\mathit{e}}
\newcommand{\eps}{\varepsilon}

\newcommand{\expect}[1]{\mathbb{E}\bracket{#1}}
\renewcommand{\var}{\variance}
\newcommand{\variance}[1]{\textrm{Var}\paren{#1}}
\newcommand{\covar}[1]{\textrm{Cov}\paren{#1}}

% Responsive Parenthesis
% \usepackage{mleftright}

\newcommand{\paren}[1]{\!\left(#1\right)}
\newcommand{\bracket}[1]{\!\left[#1\right]}

\let\oldforall\forall
\let\olexists\exists

% TODO var[] or var()

\renewcommand{\forall}{\oldforall\ }
\renewcommand{\exists}{\olexists\ }

% Custom theorem environments
\declaretheoremstyle[
  headfont=\color{bluepoli}\normalfont\bfseries,
  bodyfont=\color{black}\normalfont,
  postheadspace=\newline,
]{colored}

\declaretheoremstyle[
  headfont=\color{bluepoli}\normalfont\bfseries,
  bodyfont=\color{black}\normalfont,
  postheadspace=\newline,
]{statement}

% \setcounter{secnumdepth}{3} % % Set the depth of the section numbering
\setcounter{secnumdepth}{2} % Number up to subsections
\setcounter{tocdepth}{1} % Display up to sections in the table of contents
% \setcounter{tocdepth}{3} % Display up to subsubsections in the table of contents

% Set-up caption colors
\captionsetup[figure]{labelfont={color=bluepoli}} % Set colour of the captions
\captionsetup[table]{labelfont={color=bluepoli}} % Set colour of the captions
% \captionsetup[algorithm]{labelfont={color=bluepoli}} % Set colour of the captions

\theoremstyle{colored}
\newtheorem{theorem}{Theorem}[chapter]
\surroundwithmdframed[
  linewidth=1.5pt,
  linecolor=bluepoli,
  topline=false,
  bottomline=false,
  rightline=false,
  leftline=true,
  innertopmargin=4pt,
  innerbottommargin=4pt,
  innerrightmargin=6pt,
  innerleftmargin=6pt,
  skipabove=12pt
]{theorem}

\newtheorem{proposition}{Proposition}[chapter]
% Define a new theorem environment with a square box
\surroundwithmdframed[
  linewidth=1.5pt,
  linecolor=bluepoli,
  topline=false,
  bottomline=false,
  rightline=false,
  leftline=true,
  innertopmargin=4pt,
  innerbottommargin=4pt,
  innerrightmargin=6pt,
  innerleftmargin=6pt,
  skipabove=12pt
]{proposition}

\theoremstyle{statement}
\newtheorem{definition}[theorem]{Definition}
\newtheorem*{remark}{Remark}
\newtheorem*{remarks}{Remarks}
\newtheorem*{examples}{Examples}
\newtheorem*{example}{Example}

% Custom proof environment
\makeatletter
\renewenvironment{proof}[1][\proofname]{\par
  \pushQED{\qed}%
  \normalfont \topsep6\p@\@plus6\p@\relax
  \trivlist
  \item[\hskip\labelsep
        \color{bluepoli}\itshape
    #1\@addpunct{:}]\hskip0pt\newline\vspace{6pt}\ignorespaces
}{%
  \popQED\endtrivlist\@endpefalse
}
\makeatother

% Enhances the features of the standard "table" and "tabular" environments.
\newcommand\T{\rule{0pt}{2.6ex}}
\newcommand\B{\rule[-1.2ex]{0pt}{0pt}}

% Pseudo-code algorithm descriptions.
\newcounter{algsubstate}
\renewcommand{\thealgsubstate}{\alph{algsubstate}}
\newenvironment{algsubstates}
  {\setcounter{algsubstate}{0}%
   \renewcommand{\STATE}{%
     \stepcounter{algsubstate}%
     \Statex {\small\thealgsubstate:}\space}}
  {}

% New font size
\newcommand\numfontsize{\@setfontsize\Huge{200}{60}}

% Title format: chapter
\titleformat{\chapter}[hang]{
\fontsize{50}{20}\selectfont\bfseries\filright}{\textcolor{bluepoli} \thechapter\hsp\hspace{2mm}\textcolor{bluepoli}{|   }\hsp}{0pt}{\huge\bfseries \textcolor{bluepoli}
}

% Title format: section
\titleformat{\section}
{\color{bluepoli}\normalfont\Large\bfseries}
{\color{bluepoli}\thesection.}{1em}{}

% Title format: subsection
\titleformat{\subsection}
{\color{bluepoli}\normalfont\large\bfseries}
{\color{bluepoli}\thesubsection.}{1em}{}

% Title format: subsubsection
\titleformat{\subsubsection}
{\color{bluepoli}\normalfont\large\bfseries}
{\color{bluepoli}\thesubsubsection.}{1em}{}

% Shortening for setting no horizontal-spacing
\newcommand{\hsp}{\hspace{0pt}}

\makeatletter
% Renewcommand: cleardoublepage including the background pic
\renewcommand*\cleardoublepage{%
  \clearpage\if@twoside\ifodd\c@page\else
  \null
  \AddToShipoutPicture*{\BackgroundPic}
  \thispagestyle{empty}%
  \newpage
  \if@twocolumn\hbox{}\newpage\fi\fi\fi}
\makeatother

%For correctly numbering algorithms
\numberwithin{algorithm}{chapter}